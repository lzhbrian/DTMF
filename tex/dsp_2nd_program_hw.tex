\documentclass[UTF8,a4paper,twocolumn]{ctexart}
% \pagestyle{empty}
% \documentclass[a4paper]{article}
\usepackage{amssymb}
\usepackage{amsmath}
\usepackage{geometry}
\usepackage{graphicx}
\usepackage{float}
\usepackage{subfigure}
\usepackage{CJK}

\usepackage{lastpage}
\usepackage{titlesec}   %设置页眉页脚的宏包
\newpagestyle{main}{            
    \sethead{无42 林子恒 2014011054}{DSP --- DTMF}{page \thepage\ of \pageref{LastPage}}     %设置页眉
    %\setfoot{左页脚}{中页脚}{右页脚}      %设置页脚,可以在页脚添加 \thepage  显示页数
    \headrule                                      % 添加页眉的下划线
    \footrule                                       %添加页脚的下划线
}
\pagestyle{main}    %使用该style

\usepackage{times}
\usepackage{fancybox}
\usepackage{xcolor}
\usepackage{listings}
\lstset{breaklines}
\lstset{                        %Settings for listings package.
	language=matlab,
         numbers=left,
         numberstyle=\tiny,
         basicstyle=\tiny\ttfamily,
         stringstyle=\color{purple},
         keywordstyle=\color{purple}\bfseries,
         commentstyle=\color{olive},
       %directivestyle=\color{blue},
         frame=shadowbox,
         showspaces=false,
       %framerule=0pt,
       %backgroundcolor=\color{pink},
         rulesepcolor=\color{red!20!green!20!blue!20},
         tabsize=4,
      %rulesepcolor=\color{brown}
      %xleftmargin=5em, xrightmargin=5em, aboveskip=1em
}
\geometry{left=1cm, right=1cm, top=2cm, bottom=2cm}

  \author{无42 林子恒 2014011054} 
  \title{DSP 2nd Project --- DTMF Detection}
  \date{2016.11.25}

\CTEXsetup[format={\Large\bfseries}]{section}


%%%%%%%%%%%%%%%%%%%%%%%%%%%%%%%%%%%%
\begin{document}
  \maketitle
  \thispagestyle{empty}
	See my Github Repository for more infomation:
	
	https://github.com/lzhbrian/DTMF

\tableofcontents

\section*{File Structure}

\begin{enumerate}

\item DTMF\_1.cpp --- Main function for problem 1
\item DTMF\_2.cpp --- Main function for problem 2
\item DTMF\_3.cpp --- Main function for problem 3
\item complex.h 
\item dif\_fft.h --- FFT implementation
\item goertzel.h --- Goertzel implementation
\item find\_dtmf\_symbol.h --- judge signals
\item read\_wav.m --- Convert .wav files to .txt

\end{enumerate}

\newpage


%%%%%%%%%%%%%%%%%%%%%%%%%%%%%%%%%%%%
\section{Problem}
要求利用FFT, Goertzel算法,对给定音频文件中的双音多频信号进行检测和识别。

\begin{enumerate}

\item 下载附件包中第一小题的 10 个长度不一的音频文件,利用第一次课程设计 中编写的 FFT 程序对这 10 个文件中的 DTMF 信号进行频谱分析,最后给出 10 个文件 所对应的真实数字。
\item 编写 Goertzel 算法的 C/C++语言程序,完成(1)中的要求。
\item 下载附件包中第二小题的一个长音频文件,文件中包含了一串 DTMF 信号, 每个双音多频信号之间的时间间隔不一,对本串 DTMF 信号进行识别。

\end{enumerate}


%%%%%%%%%%%%%%%%%%%%%%%%%%%%%%%%%%%%
\section{Solution}



\subsection{Realization of DTMF using FFT}
This part is explained in the previous report. See https://github.com/lzhbrian/Fast-Fourier-Transform for more information.


\subsection{Realization of DTMF using Goertzel}

\subsection{Read the .wav files --- Convert to .txt}

\subsection{Recognition of Dataset 1 --- 10 signals}
	
	\subsubsection{FFT}
	
	\subsubsection{Goertzel}


\subsection{Recognition of Dataset 2 --- A long signal}

	\subsubsection{Judge the start-end time}


\end{document}








